\documentclass[11pt]{article}

\usepackage[margin=1in]{geometry}
\usepackage{setspace}
\onehalfspacing
\usepackage{graphicx}
\graphicspath{report_images/}
\usepackage{appendix}
\usepackage{listings}
\usepackage{float}
\usepackage{multirow}
\usepackage{amsthm}
% The next three lines make the table and figure numbers also include section number
\usepackage{chngcntr}
\counterwithin{table}{section}
\counterwithin{figure}{section}
% Needed to make titling page without a page number
\usepackage{titling}

% DOCUMENT INFORMATION =================================================
\font\titleFont=cmr12 at 11pt
\title {{\titleFont ECEN 421: Embedded Systems Design\\ North Carolina Agricultural and Technical State University \\ Department of Electrical and Computer Engineering}} % Declare Title
\author{\titleFont Chris Cannon} % Declare authors
\date{\titleFont October 1, 2018}
% ======================================================================

\begin{document}

\begin{titlingpage}
\maketitle
\begin{center}
	Homework 4
\end{center}
\end{titlingpage}

\section{Listing 4.2}
\subsection{Summary}
This program implements basic data storage and access operations. Registers 0-2 are populated, and then accessed. The stack is accessed using push and pop.
\subsection{Assembly Code}
\begin{lstlisting}
	.text
	.align 2

MemAddr .field 0x20000040; Memory address
BitBandAddr .field 0x22000810; Bit-band address

	.global main

main:

	;Data moving instructions
	MOV R0, #09h; RR0 = 0x09
	MOV R1, #11d; R1 = 0x0B
	MOV R2, #00001010b; R1 = 0x0A

	;Memory access instructions
	LDR R3, MemAddr;
	STR R2, [R3];
	STR R1, [R3, #4]!;
	LDR R0, [R3];

	PUSH {R3}
	POP {R4}

	;Bit-band operation
	LDR R5, BitBandAddr
	LDR R6, [R5]
	MOV R6, #1
	STR R6, [R5]

	;Forever loop
	B $

	.end
\end{lstlisting}
\subsection{In-Line Assembly}
\begin{lstlisting}
#include "msp.h"
void main(void)
{
    WDTCTL = WDTPW | WDTHOLD;

    __asm(" MOV R0, #09h");
    __asm(" MOV R1, #11d");
    __asm(" MOV R2, #00001010b");
    __asm(" MOV R7, #0040h");
    __asm(" MOVT R7, #2000h");
    __asm(" LDR R3, [R7]");
    __asm(" STR R2, [R3]");
    __asm(" STR R1, [R3, #4]!");
    __asm(" LDR R0, [R3]");
    __asm(" PUSH {R3}");
    __asm(" POP {R4}");
    __asm(" MOV R8, #0810h");
    __asm(" MOVT R8, #2200h");
    __asm(" LDR R5, [R8]");
    __asm(" LDR R6, [R5]");
    __asm(" MOV R6, #1"); /
    __asm(" STR R6, [R5]");

    while(1);
}
\end{lstlisting}
\subsection{C Code}
\begin{lstlisting}[language = C]
#include "msp.h"
void main(void)
{
	
    WDTCTL = WDTPW | WDTHOLD;           // Stop watchdog timer
	
     char Reg0 = 0x09;
     char Reg1 = 0x0B;
     char Reg2 = 0x0A;


     int Reg3 = 0x20000040;
     Reg3 = Reg3 + 0x00000004;
     Reg0 = Reg1;

     int Reg4 = Reg3;


     int Reg5 = 0x22000810;
     char Reg6 = 0x01;

    while(1);
}
\end{lstlisting}

\section{Listing 4.3}
\subsection{Summary}
After populating registers, this program completes basic arithmetic operations of subtraction, multiplication, and division. Then it completes logical operations of bitwise AND, OR, Exclusive Or, and Shifts.
\subsection{Assembly Code}
\begin{lstlisting}
	.text
	.align 2
	.global main

main:

	;Arithmetic operation instructions
	MOV R0, #09h; R0 = 0x09
	MOV R1, #11d; R1 = 0x0B
	MOV R2, #00001010b; R1 = 0x0A

	ADD R3, R0, R1; R3 = R0 + R1 = 0x14
	SUB R4, R2, #0002h; R4 = R2 - 0x02 = 0x08
	MOV R5, R2; R5 = R2
	MUL R5, R0; R5 = R5 * R0 = 0x5A
	SDIV R6, R5, R1; R6 = R5 / R1 = 0x08

	;Logic operation instructions
	MOV R0, #09h; R0 = 0x09
	MOV R1, #11d; R1 = 0x0B
	MOV R2, #00001010b; R1 = 0x0A

	AND R3, R0, R1; R3 = R0 + R1 = 0x14
	ORR R4, R1, #0002h; R4 = R2 - 0x02 = 0x08
	EOR R5, R0, R2; R5 = R2
	LSL R6, R0, #2; R5 = R5 * R0 = 0x5A
	LSR R7, R0, #2; R5 = R5 * R0 = 0x5A

	;Saturating arithmetic instructions
	MOV R0, #0FFFFh; R0 = 0x0000FFFF
	MOVT R0, #7FFFh; R0 = 0x7FFFFFFF
	MOV R1, #0FFFFh; R1 = 0x0000FFFF
	MOVT R1, #7FFFh; R1 = 0x7FFFFFFF

	ADDS R2, R1, R0; R2 = 0xFFFFFFFE, overflow
	QADD R3, R1, R0; R# = 0x7FFFFFFF, saturation

	;Forever loop
	B $

	.end
\end{lstlisting}
\subsection{In-Line Assembly}
\begin{lstlisting}
#include "msp.h"
void main(void)
{
	
    WDTCTL = WDTPW | WDTHOLD;           // Stop watchdog timer

    //Arithmetic operation instructions
    __asm(" MOV R0, #09h"); 
    __asm(" MOV R1, #11d"); 
    __asm(" MOV R2, #00001010b"); 

    __asm(" ADD R3, R0, R1"); 
    __asm(" SUB R4, R2, #0002h"); 
    __asm("  MOV R5, R2"); 
    __asm(" MUL R5, R0"); 
    __asm(" SDIV R6, R5, R1"); 
   //Logic operation instructions
   __asm(" MOV R0, #09h"); 
   __asm(" MOV R1, #11d"); 
   __asm(" MOV R2, #00001010b"); 

   __asm(" AND R3, R0, R1"); 
   __asm(" ORR R4, R1, #0002h"); 
   __asm(" EOR R5, R0, R2"); 
   __asm(" LSL R6, R0, #2"); 
   __asm(" LSR R7, R0, #2"); 

   //Saturating arithmetic instructions
   __asm(" MOV R0, #0FFFFh");
   __asm(" MOVT R0, #7FFFh");
   __asm(" MOV R1, #0FFFFh");
   __asm(" MOVT R1, #7FFFh"); 

   __asm(" ADDS R2, R1, R0");
   __asm(" QADD R3, R1, R0");

   while(1);
}
\end{lstlisting}
\subsection{C Code}
\begin{lstlisting}
#include "msp.h"
void main(void)
{
    WDTCTL = WDTPW | WDTHOLD;           // Stop watchdog timer
    //Arithmetic operation instructions
    int Reg0 = 0x09; 
    int Reg1 = 0x0B; 
    int Reg2 = 0x0A; 

    int Reg3 = Reg0 + Reg1; 
    int Reg4 = Reg2 - 0x02; 
    int Reg5 = Reg2; 
    Reg5 = Reg5 * Reg0; 
    int Reg6 = Reg5 / Reg1;

   //Logic operation instructions
    Reg0 = 0x09; 
    Reg1 = 0x0B; 
    Reg2 = 0x0A; 
    Reg3 = Reg0 & Reg1; 
    Reg4 = Reg2 | 0x02; 
    Reg5 = Reg0 ^ Reg2; 
    Reg6 = Reg0 << 2;
    int Reg7 = Reg0 >> 2; 

   //Saturating arithmetic instructions
    Reg0 = 0x0000FFFF; 
    Reg0 = 0x7FFFFFFF; 
    Reg1 = 0x0000FFFF; 
    Reg1 = 0x7FFFFFFF; 
    Reg2 = Reg1 + Reg0; 
    Reg3 = Reg1 + Reg0; 
    if(Reg3 == 0xFFFFFFFE)
        Reg3 = Reg1; 
   while(1);
}
\end{lstlisting}

\section{Listing 4.4}
\subsection{Summary}
This program implements loops and branching for the first time, it implements two basic loops, while and for. Within these loops, basic arithmetic is done and conditions are checked.
\subsection{Assembly Code}
\begin{lstlisting}
	.text
	.align 2
	.global main

main:

	;if else
	MOV R4, #05h; R4 = 0x05
	CMPS R4, #06h;
	BHS Greater; Branch if higher or same
	AND R4, #-1; R4 = R4 - 1
	B Done1
Greater:
	ADD R4, #1; R4 = R4 + 1
Done1:

	;for
	MOV R4, #0Ah; R4 = 0x0A
Loop1:
	ADDS R4, #-1; R4 = R4 - 1
	BPL Loop1

	;while
	MOV R4, #0006h; R4 = 0x06
	MOV R5, #0002h; R5 = 0x02
Loop2:
	ADDS R4, #-1; R4 = 0x06
	CMPS R4, R5;
	BNE Loop2

	;forever loop
	B $

	.end
\end{lstlisting}
\subsection{In-Line Assembly}
\begin{lstlisting}
#include "msp.h"
void main(void)
{
    WDTCTL = WDTPW | WDTHOLD;           // Stop watchdog timer

    //if else
    __asm(" MOV R4, #05h"); 
    int Reg4 = 0x05;
    if(Reg4 >= 0x06) 
    {
        __asm(" ADD R4, #-1"); 
    }
    else
    {
        __asm(" ADD R4, #1");
    }

   //for
    __asm(" MOV R4, #0Ah"); 
    Reg4 = 0x0A;
    for(Reg4; Reg4 > 0;  ) 
    {
        __asm(" ADDS R4, #-1"); 
        Reg4 = Reg4 - 1;
    }

   //while
    __asm(" MOV R4, #0006h"); 
    Reg4 = 0x06;
    __asm(" MOV R5, #0002h"); 
   int Reg5 = 0x02;
   while(Reg4 != Reg5)
   {
       __asm(" ADDS R4, #-1"); 
       Reg4 = Reg4 - 1;
   }

    while(1);
}
\end{lstlisting}
\subsection{C Code}
\begin{lstlisting}
#include "msp.h"

void main(void)
{
	
    WDTCTL = WDTPW | WDTHOLD;           // Stop watchdog timer

    //if else
    int Reg4 = 0x05; 
    if(Reg4 >= 0x06) 
    {
        Reg4 = Reg4 - 1; 
    }
    else
    {
        Reg4 = Reg4 + 1; 
    }

   //for
    Reg4 = 0X0A;  
    for(Reg4; Reg4 > 0; Reg--) 
    {
         
    }

   //while
    Reg4 = 0x06; 
    int Reg5 = 0x02; 

    while(Reg4 != Reg5) 
    {
        Reg4 = Reg4 - 1; 
    }

    while(1);
}
\end{lstlisting}

\section{4.5}
\subsection{Summary}
This code implements the first subroutine, or a method in C. This replace subroutine is called and executed as the code runs. It simply trades some values of registers.
\subsection{Assembly Code}
\begin{lstlisting}
	.text
	.align 2
	.global main

main:
	MOV R5, #05h; R5 = 9x95
	MOV R6, #06h; R6 = 0x06
	MOV R7, #07h; R7 = 0x07
	BL Replace;

	B $;

Replace:
	MOV R4, R7
	MOV R7, R6
	MOV R6, R5
	MOV R5, R4
	BX LR

	.end
\end{lstlisting}
\subsection{In-Line Assembly}
\begin{lstlisting}
 #include "msp.h"

void replace();

/**
 * main.c
 */
void main(void)
{
	WDT_A->CTL = WDT_A_CTL_PW | WDT_A_CTL_HOLD;		// stop watchdog timer

	// R5 = 9x95
	__asm(" MOV R5, #05h");
	// R6 = 0x06
	__asm(" MOV R6, #06h");
	// R7 = 0x07
	__asm(" MOV R7, #07h");

	replace();

	while(1);

}

void replace()
{
    __asm(" MOV R4, R7");
    __asm(" MOV R7, R6");
    __asm(" MOV R6, R5");
    __asm(" MOV R5, R4");
}
\end{lstlisting}
\subsection{C Code}
\begin{lstlisting}
#include "msp.h"

void replace();

char Reg4;
char Reg5;
char Reg6;
char Reg7;

/**
 * main.c
 */
void main(void)
{
	WDT_A->CTL = WDT_A_CTL_PW | WDT_A_CTL_HOLD;		// stop watchdog timer

	// R5 = 9x95
	Reg5 = 0x05;
	// R6 = 0x06
	Reg6 = 0x06;
	// R7 = 0x07
	Reg7 = 0x07;

	replace();

	while(1);

}

void replace()
{
    Reg4 = Reg7;
    Reg7 = Reg6;
    Reg6 = Reg5;
    Reg5 = Reg4;
}
\end{lstlisting}

\section{4.6}
\subsection{Summary}
This program works with ways to access select bits in registers such as PKHBT, which accesses certain bits from a register.
\subsection{Assembly Code}
\begin{lstlisting}
	.text
	.align 2
	.global main

main:

	;Bitfield instructions
	MOV R6, #0A12Fh; R6 = 0xA12F
	BFC R6, #8, #12; R6 = 0x2F
	SBFX R7, R6, #4, #8; R7 = 0x2

	;Packing and unpacking instructions
	MOV R8, #0AABBh; R8 = 0xAABB
	MOV R9, #0CCDh; R9 = 0xCCDD

	PKHBT R10, R8, R9, LSL #16; R9 = 0xCCDDAABB
	SXTH R11, R10, ROR #16; R10 = 0xFFFFCCDD

	;Misc instructions
	NOP
	DMB
	DSB
	MRS R12, APSR
	NOP

	.end
\end{lstlisting}
\subsection{In-Line Assembly}
\begin{lstlisting}
#include "msp.h"


/**
 * main.c
 */
void main(void)
{
	WDT_A->CTL = WDT_A_CTL_PW | WDT_A_CTL_HOLD;		// stop watchdog timer

    //Bitfield instructions
    // R6 = 0xA12F
	__asm(" MOV R6, #0A12Fh");
	// R6 = 0x2F
    __asm(" BFC R6, #8, #12");
    // R7 = 0x2
    __asm(" SBFX R7, R6, #4, #8");

    //Packing and unpacking instructions
    // R8 = 0xAABB
    __asm(" MOV R8, #0AABBh");
    // R9 = 0xCCDD
    __asm(" MOV R9, #0CCDh");

    // R9 = 0xCCDDAABB
    __asm(" PKHBT R10, R8, R9, LSL #16");
    //R10 = 0xFFFFCCDD
    __asm(" SXTH R11, R10, ROR #16");

    //Misc instructions
    __asm(" NOP \n"
          " DMB \n"
          " DSB \n"
          " MRS R12, APSR \n"
          " NOP \n");
}
\end{lstlisting}
\subsection{C Code}
\begin{lstlisting}
#include "msp.h"

volatile int *memlock;
volatile int *synclock;

void lock(volatile int *l);
void unlock(volatile int *l);

/**
 * main.c
 */
void main(void)
{
	WDT_A->CTL = WDT_A_CTL_PW | WDT_A_CTL_HOLD;		// stop watchdog timer

	*memlock = 0;
	*synclock = 0;

	//Bitfield instructions
	// R6 = 0xA12F
	char Reg6 = 0x0A12F;
	//Clear bits 8 - 20 by anding them with 0
	Reg6 & 0xFF000FF;

	//Extract bits 4 - 12 from Reg6
	char Reg7 = 0x8000FF0 & Reg6;

	//Packing and unpacking instructions
	// R8 = 0xAABB
	char Reg8 = 0xAABB;
	// R9 = 0xCCDD
	char Reg9 = 0xCCD;

	// R10 = 0xCCDDAABB
	char operandOne = 0x0000FFFF & Reg8;
	char operandTwo = 0x0000FFFF & Reg9;
	operandTwo = operandTwo << 0xF;
	char Reg10 =  operandOne & operandTwo;

	// R11 = 0xFFFFCCDD
	char Reg11 = Reg10 >> 0xF;

	// Misc instructions
	// NOP does nothing
	//Lock a memory register
	lock(*memlock);
	lock(*synclock);

	char Reg12 = ASPR;

	unlock(*memlock);
	unlock(*synclock);
}

void lock(volatile int *l)
{
    while(*l) {}
    *l = 1;
}

void unlock(volatile int *l)
{
    *l = 0;
}
\end{lstlisting}

\end{document}